\documentclass[12pt,a4paper]{article}
\usepackage[left=0.5cm, right=0.5cm, top=0.5cm, bottom=0.5cm]{geometry}
\usepackage{pgfpages}
\usepackage{amsmath, amssymb}
\usepackage{booktabs} % For tables
\usepackage{graphicx} % For images

\pgfpagesuselayout{2 on 1}[a4paper,border shrink=5mm]

\title{Physics Formula Sheet}
\author{Your Name}
\date{2023/ 2024}

\begin{document}
	\maketitle
	
	\section*{Constants}
	\begin{tabular}{lll}
		\toprule
		Constant & Symbol & Value \\
		\midrule
		Speed of light & \( c \) & \( 3.00 \times 10^8 \) m/s \\
		Gravitational constant & \( G \) & \( 6.674 \times 10^{-11} \) N(m/kg)\(^2\) \\
		Planck's constant & \( h \) & \( 6.626 \times 10^{-34} \) J.s \\
		Mass of the electron & \(m_e\) & \(9.10939 \times 10^{-31}\) kg \\
		Mass of the proton & \(m_p\) & \(1.67262 \times 10^{-27}\) kg \\
		Charge of the electron & \(-e\) & \(-1.60218 \times 10^{-19}\) C \\
		Permittivity of free space & \(\epsilon_0\) & \(8.85419 \times 10^{-12}\) C\(^2\)/J m \\
		Boltzmann constant & \(k_B\) & \(1.38066 \times 10^{-23}\) J/ K \\
		Avogadro's constant & \(N_A \) & \( 6.022 \times 10 ^ {23} \) 1/mol\\
		\bottomrule
	\end{tabular}
	
	\section*{Classical Physics}
\begin{tabular}{ll}
	\toprule
	\textbf{Title} & \textbf{Equation} \\
	\midrule
	Bragg's Reflection & \( n \lambda = 2d \sin(\theta) \) \\
	Diffraction (Single Slit) & \( \lambda = d \sin(\theta) \) \\
	Young's Double Slit & \( \frac{\Delta x}{L} = \frac{ \lambda}{d} \approxeq \sin\theta\) \\
	Heat Transfer (Fourier's Law) & \( Q = mC_v \Delta T \) \\
	Continuity Equation & \( \nabla \cdot J = - \frac{d \rho}{dt} \) \\
	Force of Gravity & \( F = G \frac{m_1 m_2}{r^2} \) \\
	Coulomb Force & \( F =  \frac{q_1 q_2}{4 \pi \epsilon_0 r^2} \) \\
	Special Relativity (Time Dilation) & \( E^2 = (pc)^2 + (m_0 c^2)^2 \) \\
	\bottomrule
\end{tabular}

	
	\section*{Nuclear and magnetic physics}
	\begin{tabular}{ll}
		{Magnetic Field} & : \(E_B = - \mu B \), \\
		& \( \mu = \frac{e}{2m} L\)\\
		& \( F_z = - \frac{\partial V}{\partial z} = \mu \frac{\partial B}{\partial z}  \)\\
		{Rigid rotator} & : \(E_\text{rot} = \frac{\textbf{L}^2}{2I}\) \\
		& \( I = \frac{m_1 m_2 }{m_1 + m_2} R^2 \) \\
		{Radioactive decay} & \( N(t) = N(0) \exp ^ {-\lambda t} = N(0) (\frac{1}{2})^{t/\tau _{1/2}}\)\\
		& \( \tau _{1/2} = \ln (2) / \lambda\)\\
	\end{tabular}
	
	\section*{Thermodynamics}
	\subsection*{Black body:}
	\begin{equation*}
		D(k) dk = \frac{\partial N(k)}{\partial k} \frac{dk}{V} = \frac{k^2}{\pi^2} dk
	\end{equation*}
	\begin{equation*}
		D(\omega) d\omega = \frac{\omega^2}{\pi^2 c^3} d\omega
	\end{equation*}
	\begin{equation*}
		u(\omega) d\omega = \frac{\omega^2}{\pi^2 c^3} k_B T d\omega \hspace*{3pt}\text{classical limit}
	\end{equation*}
		\begin{equation*}
		u(\omega) d\omega = \frac{\hbar \omega^3}{\pi^2 c^3} \frac{1}{\text{exp}(\frac{\hbar \omega}{k_B T})-1} d\omega
	\end{equation*}
	\begin{equation*}
		I(\omega) = c u(\omega) d\omega
	\end{equation*}
	
	
	
	\section*{Quantum Mechanics}
	\begin{align*}
		\text{Time-dependent Schrodinger's Equation} & : i\hbar \frac{\partial}{\partial t} \Psi (\vec{x}, t) = [-\frac{\hbar^2}{2m}(\frac{\partial ^2}{\partial x^2} + \frac{\partial ^2}{\partial y ^2} + \frac{\partial^2}{\partial z^2}) + V(x)] \\
		\text{Energy of a photon} & : E = hf \\
	\end{align*}
		\begin{align*}
		\text{Time-independent Schrodinger's Equation} & : E\phi = \hat{H}\phi = \Big(-\frac{\hbar^2}{2m}\nabla^2 + V(x) \Big)\cdot \phi \\
		\text{Energy of a photon} & : E = hf \\
	\end{align*}
	\begin{align*}
		\text{Infinite potential well} & : E_n = \frac{\hbar^2}{2m}k_n^2 = \frac{\hbar^2 \pi^2 n^2}{2mL^2} = n^2 E_0, \hspace*{5pt} \psi_n (x) = \sqrt{\frac{2}{L}} \sin (\frac{n \pi x}{L}), \hspace*{5pt} E_0 = \frac{\hbar^2 \pi^2}{2mL^2}
	\end{align*}
	\begin{align*}
		\text{Transmission through a barrier} & : T = \frac{4E(V_0 - E)}{4E(V_0 - E) + V_0^2 \sinh^2 [\sqrt{2m(V_0 - E)}\frac{l}{h}]}
	\end{align*}
	\begin{align*}
	T \approx \frac{16E(V_0-E)}{V_0^2}e^{-2\rho_2 l}, \hspace*{5pt} \text{with} \rho_2 = \sqrt{\frac{2m(V_0-E)}{\hbar^2}}, \hspace*{5pt} \rho_2 \cdot l >> 1
	\end{align*}
	\begin{align*}
		\text{De Broglie wavelength} & : \lambda = \frac{h}{m v } = \frac{h}{\sqrt{2mE}}
	\end{align*}
		\begin{align*}
		\text{Photoelectric effect} & : h\nu - \phi_0 = \frac{1}{2} m v^2 = eV
	\end{align*}
	\begin{align*}
		\text{Bohr-Sommerfeldt condition} & : \oint_C \textbf{p} \cdot d\textbf{s} = nh, \hspace*{5pt} 2\pi r = nh \text{(circular orbit)}
	\end{align*}	
	\begin{align*}
		\text{Probability current} & : j = \frac{\hbar}{2mi} (\psi ^* \frac{\partial \Psi}{\partial x} - \Psi \frac{\partial \Psi ^*}{\partial x})
	\end{align*}
	\begin{center}
	\begin{align*}
		\text{Compton scattering} & : \lambda_2 - \lambda_1 = \frac{h}{m_0 c} (1 - \cos\theta) \\
		\textbf{p}_{h\nu 1} = \textbf{p}_{h\nu 2} + \textbf{p}_e \\
		hv_1 + m_0c^2 = h\nu_2 + \sqrt{m_0^2 c^4 + p_e^2 c^2} \\
	\end{align*}
	\end{center}
	\begin{align*}
		\text{}
	\end{align*}
	
	
	
	
	
	
	
	
	
	
	
	
	
	
	
	
	
	
	
	
	
		\section*{Mathematical equations}
		\subsection*{Trigonometric functions:}
		\begin{equation}
	\int \sin^n ax dx =
	\nonumber \\ 
	-\frac{1}{a}{\cos ax} \hspace{2mm}{_2F_1}\left[
	\frac{1}{2}, \frac{1-n}{2}, \frac{3}{2}, \cos^2 ax
	\right]
\end{equation}
		\begin{equation}
		\int \sin ^2 ax dx = \frac{x}{2} - \frac{1}{4a} \sin 2ax + C
	\end{equation}
	\begin{equation}
		\int x \sin ^2 ax dx = \frac{x^2}{4} - \frac{x}{4a} \sin 2ax - \frac{1}{8a^2} \cos 2ax + C
	\end{equation}
	\begin{equation}
		\int x^2 \sin^2 x ax dx = \frac{x^3}{6} - (\frac{x^2}{4a} - \frac{1}{8a^3})\sin 2ax - \frac{x}{4a^2} \cos 2ax +C
	\end{equation}
	\begin{equation}
		\int \tan ax dx = - \frac{1}{a} \ln |\cos ax| + C = \frac{1}{a} \ln |\sec ax | +C
	\end{equation}
	\begin{equation}
		\int \frac{\cos ax}{x} dx = \ln |ax| + \sum_{1}^{\infty} (-)^k \frac{(ax)^{2k}}{2k (2k)!} +C
	\end{equation}
	\begin{equation}
		\int \cos ^2 ax dx = \frac{x}{2} + \frac{1}{4a} \sin 2ax + C
	\end{equation}
	\begin{equation}
		\int \sin^3 ax dx = \frac{\cos 3ax}{12a} - \frac{3 \cos ax}{4a} + C
	\end{equation}
	\begin{equation}
		\int \tan^2 x dx = \tan x -x +C
	\end{equation}
	\begin{equation}
		\int \sin ax \cos ax dx = - \frac{\cos^2 ax}{2a} + C
	\end{equation}
	\begin{equation}
		\int x \cos ax dx = \frac{\cos ax}{a^2} + \frac{x \sin ax}{a} + C
	\end{equation}
	\begin{equation}
		\int \cos ax dx = \frac{1}{a} \sin ax + C
	\end{equation}
	\begin{equation}
		\int x \sin ax dx = \frac{\sin ax }{a^2} -\frac{x \cos ax }{a} + C
	\end{equation}
	\begin{equation}
		\int (\sin ax) (\cos^n ax) dx = - \frac{1}{a ( n + 1)} \cos ^{n+1} ax + C
	\end{equation}
	
	
	\subsection*{Exponential functions:}

		
		
		
		
		\begin{equation}
			\int_{-\infty}^\infty e^{-ax^2}\hspace{1pt}\text{d}x = \frac{\sqrt{\pi}}{\sqrt{a}} \hspace*{3pt}(a > 0)
		\end{equation}
		\begin{equation}
			\int_{-\infty}^\infty x e^{-ax^2 + bx}\hspace{1pt}\text{d}x = \frac{\sqrt{\pi} b}{2a^{3/2}} e^{\frac{b^2}{4a}} \hspace{3pt}(\Re(a) > 0)
		\end{equation}
		\begin{equation}
			\int_{-\infty}^\infty x^n e^{-ax} \hspace{1pt} \text{d}x = \begin{cases}
				\dfrac{\Gamma (n+1)}{a^{n+1}} \hspace*{3pt} (n>-1, a>0)\\[.15in]
				\dfrac{n!}{a^{n+1}} \hspace*{3pt} (n=0,1,2,..., a>0)
			\end{cases}
		\end{equation}
		
		\begin{equation}
			\int_{-\infty}^\infty x^2 e^{-ax^2}\ {dx} = \frac{1}{2} \sqrt{\frac{\pi}{a^3}} \hspace*{3pt} (a > 0)
		\end{equation}
		\begin{equation}
			\int x e^{cx} dx = \left(\frac{x}{c}-\frac{1}{c^2}\right) e^{cx}
		\end{equation}
		\begin{equation}
			\int x^2 e^{cx} dx = \left(\frac{x^2}{c} - \frac{2x}{c^2} + \frac{2}{c^3}\right) e^{cx}
		\end{equation}
		\begin{equation}
			\int x^4 e^{-ax^2}\hspace{1pt} dx = \sqrt{\dfrac{\pi }{a}} \dfrac{3}{4a^2}
		\end{equation}
		
	\section*{Spherical coordinates}
	\begin{align*}
		x &= r \sin \theta \cos \phi \\
		y &= r \sin \theta \sin \phi \\
		z &= r \cos \phi
	\end{align*}
	
	\vspace{.1in}
	Volume fraction: 
	\[ dV = r^2 \sin \theta dr d\theta d\phi \]
	
	\vspace{.1in}
	Solid angle: 
	\[ d\Omega = \frac{dS_r}{r^2} = \sin\theta d\theta d\phi \]
	
	\vspace{.1in}
	Surface element: 
	\[ dS_r = r^2 \sin\theta d\theta d\phi \]
		
		
\begin{equation}
	\nabla f = \frac{\partial f}{\partial r} \vec{r} + \frac{1}{r} \frac{1}{r \sin \theta} \frac{\partial f}{\partial \phi} \vec{\phi}
\end{equation}

\begin{equation}
	\operatorname {div} \mathbf {F} =\nabla \cdot \mathbf {F} ={\frac {1}{r^{2}}}{\frac {\partial }{\partial r}}\left(r^{2}F_{r}\right)+{\frac {1}{r\sin \theta }}{\frac {\partial }{\partial \theta }}(\sin \theta \,F_{\theta })+{\frac {1}{r\sin \theta }}{\frac {\partial F_{\varphi }}{\partial \varphi }}.
\end{equation}
\begin{equation}
	\begin{aligned}
		\nabla \times \mathbf{F} = \frac{1}{r\sin\theta} (\frac{\partial }{\partial \theta} (A_\phi \sin \theta ) - \frac{\partial A_\theta}{\partial \phi}) \vec{r} \\
		+ \frac{1}{r}(\frac{1}{\sin\theta }\frac{\partial A_r}{\partial \phi} - \frac{\partial}{\partial r }(rA_\phi)) \vec{\theta} \\
		+ \frac{1}{r} (\frac{\partial}{\partial r}(r A_\phi) - \frac{\partial A_r}{\partial \phi}) \vec{\phi}
	\end{aligned}
\end{equation}
\begin{equation}
	\begin{aligned}
		\nabla^2 f = {1 \over r^{2}}{\partial  \over \partial r}\!\left(r^{2}{\partial f \over \partial r}\right)\!+\!{1 \over r^{2}\!\sin \theta }{\partial  \over \partial \theta }\!\left(\sin \theta {\partial f \over \partial \theta }\right)\!+\!{1 \over r^{2}\!\sin ^{2}\theta }{\partial ^{2}f \over \partial \varphi ^{2}} = \\ (\frac{\partial^2}{\partial r^2} + \frac{2}{r} \frac{\partial}{\partial r}) f
		+ \frac{1}{r^2 \sin \theta }\frac{\partial}{\partial \theta} (\sin \theta \frac{\partial}{\partial \theta})f
		+ \frac{1}{r^2 \sin ^2 \theta } \frac{\partial ^2}{\partial \phi ^2 } f
	\end{aligned}
\end{equation}


%harmonic oscillator
\section*{Harmonic oscillator:}
\begin{center}
\begin{tabular}{ccc}
	First four harmonic oscillator wavefunction & Hermite polynomials & E$_\text{n}$\\
	$\psi_0(\xi) = \left(\frac{m\omega}{\pi\hbar}\right)^{\frac{1}{4}} e^{-\frac{1}{2}\xi^2}$ & 1 & $\frac{1}{2}\hbar\omega$\\[.15in]
	$\psi_1(\xi) = \left(\frac{m\omega}{\pi\hbar}\right)^{\frac{1}{4}} \sqrt{2} \xi e^{-\frac{1}{2}\xi^2}$
	 & $2y$ & $\frac{3}{2} \hbar \omega$ \\[.15in]
	 $\psi_2(\xi) = \left(\frac{m\omega}{\pi\hbar}\right)^{\frac{1}{4}} \frac{1}{\sqrt{2}} \left(2\xi^2 - 1\right) e^{-\frac{1}{2}\xi^2}$
	 & $4y^2 - 2$ & $\frac{5}{2}\hbar\omega$\\[.15in]
	 $\psi_3(\xi) = \left(\frac{m\omega}{\pi\hbar}\right)^{\frac{1}{4}} \frac{1}{\sqrt{3}} \left(2\xi^3 - 3\xi\right) e^{-\frac{1}{2}\xi^2}$
	 & $8y^3 - 12y$ & $\frac{7}{2}\hbar\omega$\\[.15in]
\end{tabular}

\begin{tabular}{cc}
	Harmonic oscillator & $\psi_n(x) = \frac{1}{\sqrt{2^n n!}} \left(\frac{m\omega}{\pi \hbar}\right)^{\frac{1}{4}} (a^\dagger)^n e^{-\frac{1}{2}\frac{m\omega}{\hbar}x^2} \psi_0(x)$ \\[.15in]
	Raising operator & $a^\dagger = \frac{1}{\sqrt{2\hbar m\omega}} (m\omega x - ip) $ \\[.15in]
	Lowering operator & $a = \frac{1}{\sqrt{2\hbar m\omega}} (m\omega x + ip)$ \\[.15in]
	$a^\dagger |n\rangle = \sqrt{n+1}|n+1\rangle$ & $
	a |n\rangle = \sqrt{n} |n-1\rangle$\\[.15in]
	Number operator & $\hat{N} = a^\dagger a
	\hat{N} |n\rangle = n |n\rangle$\\[.15in]
	Commutation relation & $[a, a^\dagger] = aa^\dagger - a^\dagger a = 1$\\[.15in]
	Hamiltonian & $\hat{H} = \hbar \omega \left( \hat{N} + \frac{1}{2} \right)$\\[.15in]

\end{tabular}
\end{center}

\section*{Inner product and expectation}
\begin{center}
	Expectation value (discrete) \\[.15in]
	\(\langle f_i \rangle = \sum_{i} P_i f_i\) \\[.25in]
	Expectation value (continuous) \\[.15in]
	\(\langle f(x) \rangle = \int_{-\infty}^{\infty} f(x) P(x) \, dx\) \\[.15in]
	\(\langle \hat{O} \rangle = \int \psi^*(\mathbf{r}) \hat{O} \psi(\mathbf{r}) \, d^3r\) \\[.25in]
	Inner product \\[.15in]
	\(\langle \psi | \phi \rangle = \int \psi^*(x) \phi(x) \, dx\) \\[.25in]
	Variance \\[.15in]
	\(\sigma_f^2 = \langle f^2 \rangle - \langle f \rangle^2\)
\end{center}

\section*{Commutation relations}
\begin{center}
	$[A, B] = AB - BA$\\
	$[AB, C] = A[B, C] - [A, C]B$\\
	$[x, p_x] = i\hbar$\\
	$[y, p_y] = i\hbar$\\
	$[x, y] = [x, p_y] = [y, p_x] = 0$\\
\end{center}

\section*{Hydrogen atom}
\begin{center}
	Fine structure constant:\\
	$\alpha = \frac{e^2}{4\pi \varepsilon_0 \hbar c} \approx \frac{1}{137}$\\
	Bohr radius:\\
	$a_0 = \frac{\hbar}{m_e c \alpha} \approx 0.529 \times 10^{-10} \text{m}$\\
	Bohr energy:\\
	$E_n = -\frac{2\pi^2 k^2 e^4 m_e}{h^2 n^2}$\\
	Ground state energy:\\
	$E_1 = -13.6 \text{eV}$\\
	Wave function:\\
	$\psi_{n\ell m}(r, \theta, \phi) = R_{n\ell}(r) Y_{\ell m}(\theta, \phi)$\\
	Rydberg formula:\\
	$\frac{1}{\lambda} = R_H \left( \frac{1}{n_1^2} - \frac{1}{n_2^2} \right)$\\
	Rydberg constant:\\
	$R_H \approx 1.097 \times 10^7 \text{m}^{-1}$\\
	Radial wavefunctions:\\
	$R_{n\ell}(r) = N_{n\ell} r^\ell e^{-\rho/2} L^{2\ell + 1}_{n-\ell-1}(\rho)$\\
	
\end{center}


\section*{Legendre polynomials}

\section*{Angular momentum}


\section*{Hund's rule}
1: All other thing being equal, the state with the highest total spin (S), will have the lowest.\\
2: For a given spin, the state the highest total orbital angular momentum (L), consistent with overall anti-symmetrization, will have the lowest energy.\\
3: If a subshell (n, l) is no more than half filled, then the lowest energy level has J= |L-S|: if it is more than half filled, then J= L+S has the lowest energy.\\
\section*{Spin}

\section*{Clebsch-Gordan coefficients}

\section*{Free electron gas}


	
	% Add other sections/topics as needed
	
	\section*{Periodic Table}
	Insert or link to a detailed periodic table here.
	
\end{document}
